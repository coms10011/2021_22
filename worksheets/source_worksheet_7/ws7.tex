%ws6.tex
%notes for the course Mathematics for Computer Science A COMS10014
%taught at the University of Bristol
%2020_21 Conor Houghton conor.houghton@bristol.ac.uk

%To the extent possible under law, the author has dedicated all copyright 
%and related and neighboring rights to these notes to the public domain 
%worldwide. These notes are distributed without any warranty. 


\documentclass[11pt,a4paper]{scrartcl}
\typearea{12}
\usepackage{graphicx}
%\usepackage{pstricks}
\usepackage{listings}
\usepackage{color}
\usepackage{tikz}
\usetikzlibrary{decorations.markings}
\lstset{language=C}
\usepackage{fancyhdr}
\pagestyle{fancy}
\lhead{\texttt{cs-uob.github.io/COMS10014/ and github.com/coms10011/2021\_22}}
\lfoot{COMS10014 - P\&C ws7 - Conor}
\begin{document}

\section*{Probability and Combinatorics Worksheet 7}

\subsection*{Useful facts}

\begin{itemize}


\item \textbf{Integrating a polnomial}
\begin{equation}
\int x^n dx=\frac{x^{n+1}}{n+1}
\end{equation}
so the definite integral is
\begin{equation}
\int_{a}^b x^n dx=\frac{b^{n+1}}{n+1}-\frac{a^{n+1}}{n+1}
\end{equation}


\item \textbf{Integrating an exponential}
\begin{equation}
\int_{x_1}^{x_2} e^{ax} dx=\frac{1}{a}\left(e^{ax_2}-e^{ax_1}\right)
\end{equation}

\item \textbf{Integrating by parts}
  \begin{equation}
    \int_a^b udv = \left.uv\right]_a^b -\int_a^b vdu
  \end{equation}
  
  
\end{itemize}



\subsection*{Questions}

These are the questions you should make sure you work on in the workshop.

\begin{enumerate}

\item A distribution $x$ has the form
\begin{equation}
p(x)=\left\{\begin{array}{cc}x& 0\le x <1\\2-x& 1\le x<2\\0&\mbox{otherwise}\end{array}\right.
\end{equation}
What is the probability $x<1$; what is the probability $x<1.5$? What is the probability $0.5<x<1.5$? 
  
\item The distribution of tree heights in a pine tree forest is 
\begin{equation}
p(h)=\left\{\begin{array}{cc}0.3& 0\le h <2\\0.2& 2\le h<4\\0&\mbox{otherwise}\end{array}\right.
\end{equation}
What is the mean height of trees in the forest?

\item Work out the mean and variance for the distribution
  \begin{equation}
    p(x)=\left\{\begin{array}{ll}1/2a&x\in [-a,a]\\0&\mbox{otherwise}\end{array}\right.
  \end{equation}

\item Another useful distribution is the exponential distribution:
$$
p(x)=\left\{\begin{array}{cc}\lambda e^{-\lambda x}& x\ge 0\\ 0&\mbox{otherwise}\end{array}\right.
$$
What is the probability $\mbox{Prob}(x_1 < x <x_2)$ where $x_1$ and $x_2$ are both positive.

\end{enumerate}

\subsection*{Extra questions}

Do these in the workshop if you have time.

\begin{enumerate}

\item By integrating the formula for $\langle X\rangle$, what is the mean of the exponential distribution? 

\item Work out the mean of the exponential distribution by integrating
  $$1=Z=\int_0^{\infty} p(x)dx$$

\item What is the variance of the exponential distribution?
  
\end{enumerate}

\end{document}

