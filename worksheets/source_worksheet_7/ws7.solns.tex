%ws7.tex
%notes for the course Mathematics for Computer Science A COMS10014
%taught at the University of Bristol
%2021_22 Conor Houghton conor.houghton@bristol.ac.uk

%To the extent possible under law, the author has dedicated all copyright 
%and related and neighboring rights to these notes to the public domain 
%worldwide. These notes are distributed without any warranty. 


\documentclass[11pt,a4paper]{scrartcl}
\typearea{12}
\usepackage{graphicx}
%\usepackage{pstricks}
\usepackage{listings}
\usepackage{color}
\usepackage{tikz}
\usetikzlibrary{decorations.markings}
\lstset{language=C}
\usepackage{fancyhdr}
\pagestyle{fancy}
\lhead{\texttt{cs-uob.github.io/COMS10014/ and github.com/coms10011/2021\_22}}
\lfoot{COMS10014 - P\&C ws7 - Conor}
\begin{document}

\section*{Probability and Combinatorics Worksheet 7 - outline solutions}

\subsection*{Questions}

These are the questions you should make sure you work on in the workshop.

\begin{enumerate}

\item A distribution $x$ has the form
\begin{equation}
p(x)=\left\{\begin{array}{cc}x& 0\le x <1\\2-x& 1\le x<2\\0&\mbox{otherwise}\end{array}\right.
\end{equation}
What is the probability $x<1$; what is the probability $x<1.5$? What is the probability $0.5<x<1.5$?\\
\\
\\
\textbf{Solution}: The first bit is easy since clearly half the probability mass is below one and the rest is above, so $P(x<1)=0.5$, the next is harder
\begin{equation}
  P(x<1.5)=\int_0^{1.5}p(x)dx
\end{equation}
Now $p(x)$ is a complicated function but we can split this up
\begin{equation}
  P(x<1.5)=\int_0^{1}p(x)dx+\int_1^{1.5}p(x)dx=0.5+\int_1^{1.5}(2-x)dx
\end{equation}
and now we can do the integral
\begin{equation}
  P(x<1.5)=0.5+\left(2x-\frac{1}{2}x^2\right)_1^{1.5}=0.5+0.375=0.875
\end{equation}
Similarly
\begin{equation}
  P(0.5<x<1.5)=\int_{0.5}^{1.5}p(x)dx=\int_{0.5}^{1}p(x)dx+\int_1^{1.5}p(x)dx
\end{equation}
and substituting in the mass function in each subinterval
\begin{equation}
  P(0.5<x<1.5)=\int_{0.5}^{1}xdx+\int_1^{1.5}(2-x)dx=\left.\frac{x^2}{2}\right]_{0.5}^1+0.375=0.75
\end{equation}

\item The distribution of tree heights in a pine tree forest is 
\begin{equation}
p(h)=\left\{\begin{array}{cc}0.3& 0\le h <2\\0.2& 2\le h<4\\0&\mbox{otherwise}\end{array}\right.
\end{equation}
What is the mean height of trees in the forest?
\\ \\ \\
\textbf{Solution}: So 
\begin{equation}
\mu =\langle H\rangle=\int_{-\infty}^\infty hf(h)=\int_{0}^2 {0.3h}dh+\int_2^4{0.2h}dh
\end{equation}
and
\begin{equation}
\int_{0}^2 {0.3h}dh=\left.0.3\frac{h^2}{2}\right|_0^2=0.6
\end{equation}
and
\begin{equation}
\int_{2}^4 {0.2h}dh=\left.0.2\frac{h^2}{2}\right|_2^4=0.2 (8-2)=1.2
\end{equation}
so $\mu=1.8$.

\item Work out the mean and variance for the distribution
  \begin{equation}
    p(x)=\left\{\begin{array}{ll}1/2a&x\in [-a,a]\\0&\mbox{otherwise}\end{array}\right.
  \end{equation}
\\ \\ \\
\textbf{Solution}: So in this case
\begin{equation}
  \langle X\rangle =\int_{-\infty}^\infty xp(x)dx=\frac{1}{2a}\int_{-a}^a xdx=0
\end{equation}
and
\begin{equation}
  \langle X^2\rangle =\int_{-\infty}^\infty x^2p(x)dx=\frac{1}{2a}\int_{-a}^a x^2dx=\left.\frac{2x^3}{3a}\right]_{-a}^a=\frac{a^2}{3}
\end{equation}
and here the variance is the same as the second moment because the mean is zero.
  
\item Another useful distribution is the exponential distribution:
$$
p(x)=\left\{\begin{array}{cc}\lambda e^{-\lambda x}& x\ge 0\\ 0&\mbox{otherwise}\end{array}\right.
$$
What is the probability $\mbox{Prob}(x_1 < x <x_2)$ where $x_1$ and $x_2$ are both positive.
  \\ \\ \\ \textbf{Solution}: 
\begin{equation}
\mbox{Prob}(x_1 < x <x_2)=\lambda \int_{x_1}^{x_2} e^{-\lambda y}dy=-\left. e^{-\lambda y} \right]_{x_1}^{x_2}=e^{-\lambda x_1}-e^{-\lambda x_2}
\end{equation}

\end{enumerate}

\subsection*{Extra questions}

Do these in the workshop if you have time.

\begin{enumerate}

\item By integrating, what is the mean of the exponential distribution?\\
  \\
  \\ \textbf{Solution}: So
  \begin{equation}
    <X>=\lambda\int_0^\infty xe^{-\lambda x}dx
  \end{equation}
  so this is an integration by parts with $u=x$ and $dv=\exp{(-\lambda x)}dx$ so
  \begin{equation}
    du=1
  \end{equation}
  and
  \begin{equation}
    v=-\frac{1}{\lambda}e^{-\lambda x}
  \end{equation}
  giving
    \begin{equation}
    <X>=\int_0^\infty e^{-\lambda x}dx=\frac{1}{\lambda}
  \end{equation}

\item Work out the mean of the exponential distribution by differenciating
  $$1=Z=\int_0^{\infty} p(x)dx$$
  \\ \\ \\ \textbf{Solution}: Well
  \begin{equation}
    1=\lambda\int_0^\infty e^{-\lambda x}dx
  \end{equation}
  and differentiate both sides with respect to $\lambda$ to get
  \begin{equation}
    0=\int_0^\infty e^{-\lambda x}dx-\lambda \int_0^\infty xe^{-\lambda x}dx
  \end{equation}
  so
  \begin{equation}
    \langle X\rangle=\frac{1}{\lambda}
  \end{equation}

\item What is the variance of the exponential distribution?
  \\ \\ \\ \textbf{Solution}: So same again, using the formula for the mean
  \begin{equation}
    1=\lambda^2\int_0^\infty xe^{-\lambda x}dx
  \end{equation}
  and differentiate both sides with respect to $\lambda$ to get
  \begin{equation}
    0=2\lambda\int_0^\infty xe^{-\lambda x}dx-\lambda^2 \int_0^\infty x^2e^{-\lambda x}dx
  \end{equation}
  so
  \begin{equation}
    \langle X^2\rangle=\frac{2}{\lambda}
  \end{equation}
  and hence
  \begin{equation}
    \mbox{var}(X)=\frac{1}{\lambda}
    \end{equation}
\end{enumerate}

\end{document}

