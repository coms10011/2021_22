%ps1.solns.tex
%notes for the course Probability and Statistics COMS10011 
%taught at the University of Bristol
%2018_19 Conor Houghton conor.houghton@bristol.ac.uk

%To the extent possible under law, the author has dedicated all copyright 
%and related and neighboring rights to these notes to the public domain 
%worldwide. These notes are distributed without any warranty. 

\documentclass[11pt,a4paper]{scrartcl}
\typearea{12}
\usepackage{graphicx}
%\usepackage{pstricks}
\usepackage{listings}
\usepackage{color}
\lstset{language=C}
\usepackage{fancyhdr}
\pagestyle{fancy}
\lfoot{\texttt{github.com/COMS10011/2021\_22}}
\lhead{COMS10011 ws1.solns - Conor}
\begin{document}

\section*{Problem Sheet 1 - outline solutions}

\begin{enumerate}

\item In the poker hand two pair there are two pairs of cards with
  each card in the pair matched by value; the fifth card is a
  different value. What is the probability of two pairs when five
  cards are drawn randomly.\\ \\ \\ \textbf{Solution}: There are 13 choose two choices
  for the two values for the two pairs and for each pair there are
  four choose two possible cards. For the remaining card there are 11
  possible values and four possible suits. Thus, the number of
  possible pairs is
\begin{equation}
\left(\begin{array}{c}13\\2\end{array}\right)\left(\begin{array}{c}4\\2\end{array}\right)^2\times 44
=
\frac{13\times 12}{1\times 2}\times 36\times 44=123552
\end{equation}
and hence the probability is 123552/2598960=0.0475.

\item In a full house there is one pair and one
  triple, what is the probability of getting a full
  house?\\ \\ \\ \textbf{Solution}: For full house,
there are 13 possible values for the pair and 12 for the triple; including the choice of suits we have
\begin{equation}
13\times 12 \times \left(\begin{array}{c}4\\2\end{array}\right)\left(\begin{array}{c}4\\3\end{array}\right)=3744
\end{equation}
and the probability is 0.0014.


\item How many anagrams are there of the word `COVID'?\\ \\ \\
  \textbf{Solution}: The number of ways to reorder five letters is $5!=5\times 4\times 3\times 2=120$, so, if you don't count the word itself, there are 119 anagrams.


\item How many distinct anagrams are there of the word
  `CUMMINGS'?\\ \\ \\ \textbf{Solution}: This one is harder because of
  the repeated letter, if you just look at the reorderings that gives
  $8!$ but that double counts each word since you can swap the `M's,
  so the answer, leaving out the word itself, is $8!/2
  -1=20159$. Another way to think about it that allows you to
  generalize to more complicated examples is to think that there are eight slots, two are allocated to `M's and one to each of the other letters, so $n$ the number of anagrams is
  \begin{equation}
    n=\left(\begin{array}{c}8\\2,1,1,1,1,1,1\end{array}\right)-1=20159
  \end{equation}
  


\end{enumerate}

\subsection*{Extra questions}

\begin{enumerate}


\item When it started in 1987 the Irish lottery has 36 numbers;
  participants paid 50 Irish pence to buy a combination of six
  different numbers; they would win if these numbers matched the six
  drawn. In the last week in May in 1992 a syndicate tried to buy all
  combinations of numbers. If they had succeeded how many numbers
  would they have bought?\\ \\ \\ \textbf{Solution}: Well this is just 36 choose six:
\begin{equation}
\left(\begin{array}{c}36\\6\end{array}\right)=1947792
\end{equation}
so they would've spend $973,896$ Irish pounds.


\item From a group of three undergraduates and five graduate students,
  four students are randomly selected to act as TAs. What is the
  chance there will be exactly two undergraduate
  TAs?\\ \\ \\ \textbf{Solution}: So this is another counting exercise,
  the total ways of selecting four out of eight is
\begin{equation}
\left(\begin{array}{c}8\\4\end{array}\right)=frac{8\times 7\times 6 \times 5}{1\times 2\times 3\times 4}
=70
\end{equation}
Now the number of ways of choosing two undergraduates out of three is
three and the number of ways of picking two graduates out of five is
10. Hence the answer is $3/7$.

\item Prove
\begin{equation}
\left(\begin{array}{c}n\\r\end{array}\right)=\left(\begin{array}{c}n\\n-r\end{array}\right)
\end{equation}
\\ \\ \\ \textbf{Solution}: This follows from the definition
\begin{equation}
\left(\begin{array}{c}n\\r\end{array}\right)=\frac{n!}{r!(n-r)!}
\end{equation}
which stays the same if you swap $r$ and $n-r$.

\item How many distinct anagrams has the word `OROONOKO'?
  \\ \\ \\
  \textbf{Solution}: So there are eight letters, five of which are `O', so following the answer to `CUMMINGS' the anwer is
  \begin{equation}
    n=\left(\begin{array}{c}8\\5,1,1,1\end{array}\right)-1=8\times 7\times 6-1=335
  \end{equation}
  
\end{enumerate}
  
\end{document}

