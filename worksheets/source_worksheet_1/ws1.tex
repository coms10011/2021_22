%ps1.tex
%notes for the course Probability and Statistics COMS10011 
%taught at the University of Bristol
%2019_20 Conor Houghton conor.houghton@bristol.ac.uk

%To the extent possible under law, the author has dedicated all copyright 
%and related and neighboring rights to these notes to the public domain 
%worldwide. These notes are distributed without any warranty. 

\documentclass[11pt,a4paper]{scrartcl}
\typearea{12}
\usepackage{graphicx}
%\usepackage{pstricks}
\usepackage{listings}
\usepackage{color}
\lstset{language=C}
\usepackage{fancyhdr}
\pagestyle{fancy}
\lhead{\texttt{cs-uob.github.io/COMS10014/ and github.com/coms10011/2021\_22}}
\lfoot{COMS10014 - P\&C ws1 - Conor}
\begin{document}

\section*{Probability and Combinatorics Worksheet 1}

\subsection*{Useful facts}

\begin{itemize}

\item The \textbf{power set}, $\mathcal{P}(A)$, is the set of subsets of $A$:
  \begin{equation}
    \#(\mathcal{P}(A))=2^{\#(A)}
  \end{equation}

\item\textbf{Factorial}: the number of ways to order $n$ elements is
  \begin{equation}
    n!=n\times(n-1)\times\ldots\times2\times1
  \end{equation}
  
\item \textbf{Combinations}: The number of ways of choosing $r$ items out of $n$ is 
\begin{equation}
\left(\begin{array}{c}n\\r\end{array}\right)=\frac{n(n-1)(n-2)\ldots(n-r+1)}{r(r-1)(r-2)\ldots 1}=\frac{n!}{r!(n-r)!} 
\end{equation}


\item \textbf{Combinations}: The number of ways of splitting $n$ items into sets of size $r_1$, $r_2$ through to $r_k$ with
  \begin{equation}
    r_1+r_2+\ldots+r_k=n
  \end{equation}
  is
\begin{equation}
\left(\begin{array}{c}n\\r_1,r_2,\ldots,r_k\end{array}\right)=\frac{n!}{r_1!r_2!\ldots r_k!} 
\end{equation}
 
\item \textbf{Cards}: 52 cards made up of four suits; in each suit
  there are 13 values, ace, two through to ten and the jack, queen,
  king.

\item \textbf{Poker hands}: the number of poker hands is 
\begin{equation}
\left(\begin{array}{c}52\\5\end{array}\right)= 2598960
\end{equation}

\end{itemize}


\subsection*{Questions}

These are the questions you should make sure you work on in the workshop.

\begin{enumerate}

\item In the poker hand \textsl{two pair} there are two pairs of cards
  with each card in the pair matched by value; the fifth card has a
  different value to either pair. What is the probability of two pairs
  when five cards are drawn randomly.


\item In a \textsl{full house} there is one
  pair and one triple, what is the probability of getting a full
  house?

\item How many anagrams are there of the word `COVID'?

\item How many distinct anagrams are there of the word `CUMMINGS'?
  
\end{enumerate}

\subsection*{Extra questions}

These are extra questions you might attempt in the workshop or at a later time.

\begin{enumerate}

\item When it started in 1987 the Irish lottery has 36 numbers;
  participants paid 50 Irish pence to buy a combination of six
  different numbers; they would win if these numbers matched the six
  drawn. In the last week in May in 1992 a syndicate tried to buy all
  combinations of numbers. If they had succeeded how many numbers
  would they have bought? In the event the lottery shut down lots of
  the lottery machines so they only bought most of the numbers, they
  nonetheless had the winning number but shared the prize three
  ways. However, because of the roll-over prize and the match-5 and
  match-4 prizes, they are thought to have made a substantial
  profit. The lottery was redesigned after this to have more numbers.


\item From a group of three undergraduates and five graduate students,
  four students are randomly selected to act as TAs. What is the
  chance there will be exactly two undergraduate TAs?

\item Prove
\begin{equation}
\left(\begin{array}{c}n\\r\end{array}\right)=\left(\begin{array}{c}n\\n-r\end{array}\right)
\end{equation}

\item How many distinct anagrams has the word `OROONOKO'?


\end{enumerate}

\end{document}

