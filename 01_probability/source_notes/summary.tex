
\ifind
\section*{Summary}
\else
\subsection*{1 Probability theory}
\fi

\begin{itemize}
\item A \textbf{sample space} is a set of point, they are the possible \textbf{outcomes} of a \textbf{trial}.
\item An \textbf{event} is a subset of a sample space.
\item A \textbf{probability} is a map from events to real numbers such that
  \begin{enumerate}
    \item $P(A)\ge 0$ for all events.
    \item $P(X)=1$
    \item If $A\cap B=\emptyset$ for two events $A$ and $B$ then 
      \begin{equation}
        P(A\cup B)=P(A)+P(B)
      \end{equation}
\end{enumerate}
\item A \textbf{probability mass function} is a map from points in the sample space to real numbers such that
  \begin{enumerate}
\item $p(x)\ge 0$ for all $x\in X$
\item $\sum_{x\in X} p(x)=1$
  \end{enumerate}
\item $P(A)=\sum_{x\in A}p(x)$
\item If all the points in a sample space have the same probability then
  \begin{equation}
P(A)=\frac{\mbox{number of points in }A}{\mbox{number of points in }X}=\frac{\#A}{\#{X}}
  \end{equation}
  where $\#(A)$ means the number of points in $A$.
  \item The \textbf{binomial coefficient}
\begin{equation}
\left(\begin{array}{c}n\\r\end{array}\right)=\frac{n!}{r!(n-r)!}
\end{equation}
counts the number of subsets of size $r$ in a set of $n$ objects and 
\begin{equation}
n!=n\times (n-1)\times (n-2)\times \ldots \times 2 \times 1
\end{equation}
\item The \textbf{partition function}
\begin{equation}
\left(\begin{array}{c} n\\n_1,n_2,\ldots,n_r\end{array}\right)=\frac{n!}{n_1!n_2!\ldots n_r!}
\end{equation}
where $n_1+n_2+\ldots+n_r=n$ counts the number of ways a set of $n$
objects can be split up into $r$ subgroups of sizes $n_1$, $n_2$ and
so on to $n_r$.

\end{itemize}

